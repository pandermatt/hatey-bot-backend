\begin{itemize}
    \item What model did I train? - Classifier, Transformer
    \item Why did I choose this kind of model?
\end{itemize}

\begin{tcolorbox}[colback=green!5!white,colframe=green!75!black,title=Example]
  I trained a linear regression model with scikit-learn.
  I think this model is well-suited for a regression task and it is simple to interpret what the model learns.
\end{tcolorbox}


We trained several models and tested them on the \textit{Toxic Comment Classification Challenge}~\cite{jigsaw-toxic-comment-classification-challenge} dataset.
The best performing model was the Transformer model, which we used as a baseline for our experiments.
We then trained several ensemble classifiers on the same dataset and compared their performance.
The best performing ensemble classifier was the ExtraTreesClassifier, which we used as our final model in combination with the Transformer model.
The predictions of all models are displayed on the website, given an input comment.


\paragraph{Transformer}
For the task of segmentation analysis we used the pretrained transformer model \textit{cardiffnlp/twitter-roberta-base-sentiment} \cite{barbieri2020tweeteval}.
The model was evaluated on the \textit{glue/sst2} dataset also known as Stanford Sentiment Treebank \cite{socher2013recursive}.
 The Dataset consist of a two-way (positive/negative) human annotated sentiments for movie reviews. Although our model was trained
 on tweets annotated with the labels \textit{positive}, \textit{negative} and \textit{neutral}, we decided to use the \textit{glue/sst2} dataset for evaluation,
 to evaluate it also on larger text corpora. Without fine-tuning the model achieved an accuracy of 0.87 on the \textit{glue/sst2} dataset if we just
 ignored the \textit{neutral} label and took the one with the next highest probability as the prediction.

 % create a figure
\begin{figure}[h]
\centering
\includegraphics[width=0.5\textwidth]{figures/transformer.png}
\caption{Transformer model}
\label{fig:transformer}
\end{figure}

Transformer models are very powerful and can be used for many different tasks. We decided to use a additonal transformer model
for the task of toxicity analysis. We used the pretrained model \textit{roberta-base} \cite{liu2019roberta}. We fine-tuned the model on the
\textit{ConvAbuse} dataset \cite{cercas-curry-etal-2021-convabuse}. This dataset contains the labels ableism, homophobic,
intellectual, racist, sexist, sex_harassment and transphobic.

% create one figure containing two subfigures
\begin{figure}[h]
\begin{subfigure}[b]{0.5\textwidth}
  \centering
  \includegraphics[width=\textwidth]{figures/transformer_toxicity.png}
  \caption{Transformer model}
  \label{fig:transformer_toxicity}
\end{subfigure}
\begin{subfigure}[b]{0.5\textwidth}
  \centering
  \includegraphics[width=\textwidth]{figures/transformer_toxicity_confusion_matrix.png}
  \caption{Confusion matrix}
  \label{fig:transformer_toxicity_confusion_matrix}
\end{subfigure}
\caption{Transformer model for toxicity analysis}
\label{fig:transformer_toxicity}
\end{figure}


\paragraph{Classifier}

We used the BaggingClassifier, ExtraTreesClassifier, RandomForestClassifier, AdaBoostClassifier and GradientBoostingClassifier.
